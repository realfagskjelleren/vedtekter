\section{Styret} \label{styret}

\subsection{Styret} \label{subsec:styret}
\begin{enumerate}
    \item Styret skal forvalte Realfagskjelleren og sørge for forsvarlig drift. 
    \item Styret består av styrets leder, nestleder, kasserer (daglig leder), innkjøpsansvarlig,  SoMe-ansvarlig og en kjellerkontakt fra hver av linjeforeningene (jf. \ref{organisasjon}) i tillegg til avtroppende styreleder og kasserer frem til overtakelseskontrakt er signert (jf. \ref{overtakelse}). 
    \item Styret har styremøter så ofte styreleder eller styret finner dette hensiktetmessig og i tråd med vedtektene og styringsdokumentet.
    \item Styret er beslutningsdyktig når et flertall av styrets medlemmer er tilstede. Vedtak av saker i styret gjøres ved absolutt 50\% flertall.
    \item Styret skal forberede generalforsamlinger, herunder også behandle vedtektsendringer og fremlegge årsregnskap og budsjett.
    \item Styret har fullmakt til å velge en revisor til å revidere årsregnskapet når de anser det som hensiktsmessig. Styret plikter seg til å informere sine medlemmer og revisjonsberetningen, og dersom 1/3 av medlemmene stiller mistillit til forholdene rundt revisjonen kan det kalles inn til ekstraordinær generalforsamling.
    \item Permanent frafall:
    \begin{enumerate}
        \item Ved permanent frafall fra leder, nestleder eller kasserer skal det kalles inn til ekstraordinær generalforsamling, og det skal holdes valg for de aktuelle stillingene.
        \item Ved permanent frafall fra en av linjeforeningens kjellerkontakter, SoMe-ansvarlig eller innkjøpsansvarlig kan styret supplere seg selv. Dersom det mangles mer enn en kjellerkontakt skal det kalles inn til ekstraordinær generalforsamling og holdes et valg for de aktuelle stillingene.
    \end{enumerate}
\end{enumerate}

\subsection{Overtakelse}\label{overtakelse}
Styret trer inn to (2) uker etter generalforsamling. I de to (2) ukene etter generalforsamlingen, skal det være minst to (2) overføringsseminarer mellom det gamle og det nye styret.
\begin{enumerate}
    \item Den påtroppende styrelederen trer først inn som nestleder og resterende valgte medlemmer vil tre inn som styremedlemmer. Påtroppende styreleder og kasserer skal umiddelbart etter generalforsamling forsøke å gjennomføre "Ansvarlig Vertskap", "Kunnskapsprøven i alkoholloven" og "Etablererprøven" (heretter "de nødvendige prøvene"). Avtroppende styreleder og kasserer vil beholde rollene sine frem til dette.
    \item Etter både påtroppende styreleder og kasserer har bestått de nødvendige prøvene skal de senest syv (7) dager etter den sist beståtte prøven signere overtakelseskontrakt med avtroppende styreleder og kasserer. Da trer de inn i sine respektive roller og den valgte nestlederen fra generalforsamling trer inn som nestleder. I tillegg trer tidligere leder og kasserer ut av styret. \label{overtakelseskontrakt}
    \item Etter at overtakelseskontrakten er signert skal dette informeres om til medlemmene av Realfagskjelleren gjennom minst en (1) av de primære informasjonskanalene samme dag.
    \item Dersom både påtroppende styreleder og kasserer har bestått de nødvendige prøvene før generalforsamling vil styret tre direkte inn i de valgte rollene to (2) uker etter generalforsamling. Det skal likevel signeres overtakelseskontrakt senest syv (7) dager etter dette.
    \item Dersom en (1) eller begge påtroppende ikke har bestått de nødvendige prøvene innen 15. september skal det kalles inn til ekstraordinær generalforsamling hvor det vurderes gjenvalg på rollen(e) det gjelder.
    \begin{enumerate}
        \item Dersom kun en (1) av de påtroppende har bestått de nødvendige prøvene innen fristen kan vedkommende kreve å tre inn i sin rolle senest 22. september. Dersom vedkommende trer inn som leder skal det signeres overtakelseskontrakt (jf. §5-2.\ref{overtakelseskontrakt}).
    \end{enumerate}

        
\end{enumerate}

\subsection{Signaturrett og prokurarett} \label{signaturrett og prokurarett}

Styrets leder og kasserer (daglig leder) har signaturrett i fellesskap.
Styrets leder, nestleder og kasserer (daglig leder) har prokura hver for seg.
Styret kan ytterligere tildele prokura ved behov.

\subsection{Styringsdokumenter} \label{styringsdokumenter}

Styret skal bruke styringsdokumenter som sedvanerett for driften av linjeforeningen.
Hver styrestilling skal ha sitt eget styringsdokument.
For å bryte, endre eller opprette et styringsdokument kreves det absolutt $2/3$ flertall i styret.
Styringsdokumenter kan også brytes, endres eller opprettes av generalforsamling ved alminnelig flertall. 
Styringsdokumentene skal ikke stå i konflikt med vedtektene.
Styret kan vedta endringer av styringsdokumentene, men endringene skal vedtas offisielt på neste ordinære generalforsamling.

\subsection{Kasserer} \label{kasserer}

Kasserer innehar rollen som daglig leder og skal forholde seg til sitt styringsdokument.
Kasserer plikter å legge frem en årsberetning for foreningens økonomi og årsregnskap, samt budsjett under ordinær generalforsamling.
Kasserer kan overstyres i økonomiske avgjørelser ved absolutt $2/3$ flertall i styret. 
Generalforsamlingen overstyrer økonomiske avgjørelser med alminnelig flertall.

\subsection{Mistillitsforslag} \label{mistillitsforslag}

Ethvert medlem kan stille mistillitsforslag til et medlem av styret, der resterende medlemmer i styret behandler forslaget.
Styret kan suspendere den anklagde dersom dette er hensiktsmessig.
Stilles det mistillitsforslag mot flere enn halvparten i styret skal dette behandles på ekstraordinær generalforsamling.
Anklagde har rett til å forsvare seg og få innsyn i anklagen.

\subsection{Utestengelse} \label{utestengelse}

Styret kan utestenge medlemmer ved ekstraordinære forhold som kan skade Realfagskjelleren.
Dette krever enstemmig flertall, der den utestengte kan anke avgjørelsen til generalforsamlingen.