\section{Medlemskap} \label{medlemskap}

\subsection{Opptak og medlemskap} \label{opptak og medlemskap}

Et medlem regnes som frivillig i Realfagskjelleren, og det er ingen medlemskontingent.
Et medlem er tatt opp for en periode på 2 år, og medlemmet kan si opp medlemskapet sitt når som helst ved å kontakte styret.
Styret kan, ved forespørsel, utvide et medlems medlemskap årevis.

Styret har opptak av nye medlemmer ved behov, for å sikre at enhver som ønsker å bli medlem slutter seg til Realfagskjelleren sitt formål.

Dersom et medlem blir utestengt eller kastet ut av en av Realfagskjellerens tilhørende linjeforeninger plikter medlemmet seg til å melde hendelsen til styret.

\subsection{Fordeling av medlemsmasse} \label{fordeling av medlemsmasse}

Det skal jobbes for en lik fordeling i Realfagskjellerens medlemmer fra hver linjeforening som er nevnt i \ref{organisasjon}.

\subsection{Inhabilitet} \label{inhabilitet}

Ingen medlemmer i Realfagskjelleren, har stemmerett og/eller beslutningsrett utenfor Generalforsamlingen når en er inhabil. 
Et medlem i Realfagskjelleren er inhabil når vedkommende selv, nær familie og/eller nære venner har egeninteresse i den beslutningen som skal fattes, eller når det foreligger andre særegne forhold som er egnet til å svekke tilliten til medlemmets upartiskhet.