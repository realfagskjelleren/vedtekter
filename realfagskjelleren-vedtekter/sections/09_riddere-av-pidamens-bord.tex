\section{Riddere av Pi-damens bord} \label{riddere av pi-damens bord}

\subsection{Grunnlag for ordenen} \label{grunnlag for ordenen}

Personer som har utmerket seg eksepsjonelt innen Realfagskjellerens drift og virke kan utnevnes som Ridder av Pi-damens Bord.
Personen skal ha gjennomført et arbeid som langt forbigår det som er forventet i deres stilling, og har vært definerende for Realfagskjellerens drift og virke.
Ordenen skal fungere som en kilde til råd for Realfagskjellerens medlemmer.

\subsection{Nominasjonsprosess} \label{nominasjonsprosess}

Riddere nomineres av  et utvalg bestående av tre (3) personer.
Deres jobb er å behandle og vurdere innsendte nominasjoner.
Utvalget sitter fra ordinær generalforsamling frem til neste ordinære generalforsamling.
Utvalget skal bestå av leder, et medlem av Realfagskjelleren og et tidligere eller nåværende medlem av Realfagskjelleren 
De to (2) sistnevnte skal velges på generalforsamling.
Utvalget skal behandle innsendte nominasjoner og stemme over dem.
En person kan nomineres til ridder så lenge de er tatt opp i Realfagskjelleren og ikke sitter i nåværende styre.
En god nominasjonstekst skal være en grundig redegjørelse som utdyper hva kandidaten har gjort for Realfagskjelleren for å fortjene å bli slått til ridder. 
En kandidat har ikke anledning til å nominere seg selv.

\subsection{Frister} \label{frister}

Styret har frem til fire (4) uker etter semesterstart på våren for å åpne for nominasjoner.
Deretter har utvalget to (2) uker på å behandle nominasjoner.
De nye ridderne skal senest annonseres på ordinær generalforsamling.